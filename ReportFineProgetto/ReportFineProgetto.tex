\documentclass[a4paper]{report}
\usepackage{graphicx}
\usepackage{eso-pic}
\usepackage{hyperref}
\hypersetup{
	colorlinks=true,
	linkcolor=blue,
	filecolor=magenta,
	urlcolor=blue,
	pdftitle={Report Finale "Laboratorio Integrato"},
	pdfpagemode=FullScreen,
}

\usepackage[italian]{babel}


\newcommand\BackgroundPic{%
	\put(0,0){
		\parbox[b][\paperheight]{\paperwidth}{%
			\vfill
			\centering
			\includegraphics[width=\paperwidth,height=\paperheight,%
			keepaspectratio]{its-cover.pdf}%
			\vfill
		}
	}
}
\author{De lazzari, Dellera, Oglietti,\\
 Murta, Cafasso, Carrieri,\\
 Zuccarella}
\title{Report Finale "Laboratorio Integrato"\\
\large Gruppo 6, Cloud Fiesta}
\date{\today}
\makeindex

\begin{document}
\AddToShipoutPicture*{\BackgroundPic}
\maketitle
\tableofcontents
\chapter{Introduzione}
\author{Riccardo Oglietti} 
	\section{Il Progetto}
	Il qui presente report ha lo scopo di illustrare lo svolgimento nella sua interezza lo svolgimento del progetto a
	opera del gruppo "Cloud Fiesta", il progetto e' stato commissionato dai docenti Blanchietti Andrea e Zimuel Enrico
	nell'ambito del corso \textit{"Laboratorio Integrato"}.

	Lo scopo del progetto e' quello di realizzare una piattaforma di \emph{e-commerce} per conto di un azienda che si
	occupa di commercio al dettaglio, il sistema deve essere \emph{scalabile} in maniera da poter limitare i costi a
	quanto strettamente necessrio e potersi mantenere aderente con le esigenze di crescita aziendale.
	Inoltre, e' essenziale che la piattaforma possa avere degli \emph{standard di sicurezza elevati}, come ben sappiamo,
	durante i recenti anni si e' verificata un impennata dei crimini legati alla \textit{Cybersecurity}, con un
	particolare aumento durante la corrente pandemia da COVID-19, come illustrato
	\href{https://www.interpol.int/en/News-and-Events/News/2020/INTERPOL-report-shows-alarming-rate-of-cyberattacks-during-COVID-19}{dall'Interpol}.
	E' quindi fondamentale che un'applicazione che gestisce flussi di denaro sia quindi estremamente solida dal punto di
	vista della sicurezza informatica.
	Una seconda sezione del progetto, prevede che ogni gruppo si occupi di eseguire dei \emph{penetration test} sul
	gruppo dall'\emph{ID} successivo. Questo per simulare l'ingaggio di un azienda esterna allo scopo di testare la
	sicurezza di un prodotto prima di rilasciarlo effettivamente sul mercato, uno step di decisiva importanza che
	permettera' ai componenti di ogni gruppo di sperimentare le proprie conoscenze di sicurezza informatica all'interno
	di una situazione altamente realistica.

	Vista la complessita' del progetto, e' stato scelto di realizzarlo tramite Team multidisciplinari, con componenti
	appartenenti ad due corsi afferenti agli indirizzi di \emph{Cloud Specialist} e \emph{ICT Security Specialist}.
	All'interno di questi due corsi sono presenti le competenze tecniche atte a svolgere il progetto commissionato,
	coprendo sia l'area di sicurezza e di architettura della rete interna, che quella di utilizzo delle piattaforme
	cloud che permettono di assicurare la scalabilita' necessaria all'azienda.

	\section{Il Team}
	Gli stutenti di entrambi i corsi sono stati divisi in sei differenti gruppi, composti da un totale di otto persone,
	il nostro gruppo, denominato "\emph{Cloud Fiesta}" e' composto dai seguenti studenti:
	\begin{itemize}
		\item \textbf{Cafasso Giovanni}
		\item \textbf{Carrieri Riccardo}
		\item \textbf{De Lazzari Riccardo}
		\item \textbf{Dellera Lorenzo}
		\item \textbf{Murta Alessio}
		\item \textbf{Oglietti Riccardo}
		\item \textbf{Zuccarella Andrea}
	\end{itemize}
	Suddivisi rispettivamente all'interno dei due corsi come da tabella:
	\begin{center}
		\begin{tabular}{c|c}
			Cloud Specialist & ICT Security Specialist \\
			\hline
			Cafasso Giovanni & De Lazzari Riccardo \\
			Carrieri Riccardo & Dellera Lorenzo \\
			Murta Alessio & Oglietti Riccardo \\
			Zuccarella & \\
		\end{tabular}
	\end{center}

\end{document}
