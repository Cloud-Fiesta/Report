\documentclass[a4paper]{report}
\usepackage{graphicx}
\usepackage{eso-pic}
\usepackage{hyperref}
\hypersetup{
	colorlinks=true,
	linkcolor=blue,
	filecolor=magenta,
	urlcolor=blue,
	pdftitle={Report Finale "Laboratorio Integrato"},
	pdfpagemode=FullScreen,
}

\usepackage[italian]{babel}


\newcommand\BackgroundPic{%
	\put(0,0){
		\parbox[b][\paperheight]{\paperwidth}{%
			\vfill
			\centering
			\includegraphics[width=\paperwidth,height=\paperheight,%
			keepaspectratio]{its-cover.pdf}%
			\vfill
		}
	}
}
\author{De lazzari, Dellera, Oglietti,\\
 Murta, Cafasso, Carrieri}
\title{Report Finale "Laboratorio Integrato"\\
\large Gruppo 6, Cloud Fiesta}
\date{\today}
\makeindex

\begin{document}
\AddToShipoutPicture*{\BackgroundPic}
\maketitle
\tableofcontents
\chapter{Introduzione}
\author{Riccardo Oglietti} 
	\section{Il Progetto}
	Il qui presente report ha lo scopo di illustrare lo svolgimento nella sua interezza lo svolgimento del progetto a
	opera del gruppo "Cloud Fiesta", il progetto e' stato commissionato dai docenti Blanchietti Andrea e Zimuel Enrico
	nell'ambito del corso \textit{"Laboratorio Integrato"}.

	Lo scopo del progetto e' quello di realizzare una piattaforma di \emph{e-commerce} per conto di un azienda che si
	occupa di commercio al dettaglio, il sistema deve essere \emph{scalabile} in maniera da poter limitare i costi a
	quanto strettamente necessrio e potersi mantenere aderente con le esigenze di crescita aziendale.
	Inoltre, e' essenziale che la piattaforma possa avere degli \emph{standard di sicurezza elevati}, come ben sappiamo,
	durante i recenti anni si e' verificata un impennata dei crimini legati alla \textit{Cybersecurity}, con un
	particolare aumento durante la corrente pandemia da COVID-19, come illustrato
	\href{https://www.interpol.int/en/News-and-Events/News/2020/INTERPOL-report-shows-alarming-rate-of-cyberattacks-during-COVID-19}{dall'Interpol}.
	E' quindi fondamentale che un'applicazione che gestisce flussi di denaro sia quindi estremamente solida dal punto di
	vista della sicurezza informatica.
	Una seconda sezione del progetto, prevede che ogni gruppo si occupi di eseguire dei \emph{penetration test} sul
	gruppo dall'\emph{ID} successivo. Questo per simulare l'ingaggio di un azienda esterna allo scopo di testare la
	sicurezza di un prodotto prima di rilasciarlo effettivamente sul mercato, uno step di decisiva importanza che
	permettera' ai componenti di ogni gruppo di sperimentare le proprie conoscenze di sicurezza informatica all'interno
	di una situazione altamente realistica.

	Vista la complessita' del progetto, e' stato scelto di realizzarlo tramite Team multidisciplinari, con componenti
	appartenenti ad due corsi afferenti agli indirizzi di \emph{Cloud Specialist} e \emph{ICT Security Specialist}.
	All'interno di questi due corsi sono presenti le competenze tecniche atte a svolgere il progetto commissionato,
	coprendo sia l'area di sicurezza e di architettura della rete interna, che quella di utilizzo delle piattaforme
	cloud che permettono di assicurare la scalabilita' necessaria all'azienda.
	\section{Il Team}
	Gli stutenti di entrambi i corsi sono stati divisi in sei differenti gruppi, composti da un totale di otto persone,
	il nostro gruppo, denominato "\emph{Cloud Fiesta}" e' composto dai seguenti studenti:
	\begin{itemize}
		\item \textbf{Cafasso Giovanni}
		\item \textbf{Carrieri Riccardo}
		\item \textbf{De Lazzari Riccardo}
		\item \textbf{Dellera Lorenzo}
		\item \textbf{Murta Alessio}
		\item \textbf{Oglietti Riccardo}
	\end{itemize}
	Suddivisi rispettivamente all'interno dei due corsi come da tabella:
	\begin{center}
		\begin{tabular}{c|c}
			Cloud Specialist & ICT Security Specialist \\
			\hline
			Cafasso Giovanni & De Lazzari Riccardo \\
			Carrieri Riccardo & Dellera Lorenzo \\
			Murta Alessio & Oglietti Riccardo \\
			Zuccarella & \\
		\end{tabular}
	\end{center}
	Come consigliato dai docenti, sono stati assegnati alcuni \emph{ruoli} in grado di aiutarci con l'organizzazione
	delle mansioni e in genere della gestione del progetto, in particolare abbiamo individuato il ruolo di \emph{Team
	Leader} e di e di \emph{Co-Team Leader}, essi sono stati rispettivamente assegnati a \emph{Oglietti Riccardo} e
	\emph{Murta Alessio}. Abbiamo optato per assegnare queste due cariche ripartendole tra i due differenti corsi che
	compongono il gruppo in maniera da manternere un buon livello di equita' e rappresentanza per entrambe le anime del
	team.  Da notare infine come lo studente "Zuccarella Andrea" si e' presentato a un singolo incontro durante la
	stesura dell'intero progetto, e non ha effettivamente contribuito in alcuna maniera allo svolgimento dello stesso.

\chapter{Strumenti tecnico-organizzativi}
%\author{Oglietti Riccardo} mettere tutti gli autori?
	\section{GANTT e cronoprogramma}
	Innanzitutto parlando di strumenti tecnico-organizzativi non e' possibile iniziare senza descrivere il
	"\emph{GANTT}.  Strumento principe per l'organizzazione delle tempistiche, si tratta di una tabella a doppia entrata
	che permette di assegnare alcuni \emph{task} ritenuti fondamentali a un membro e un momento nel quale realizzarlo.

	Ecco una lista riassuntiva dei processi e degli \emph{step} fondamentali che abbiamo individuato al fine della
	realizzazione ottimale del progetto, divisi in base al corso di afferenza dei destinatari:
	\begin{itemize}
		\item \begin{enumerate}
				\item Parsing file CSV
				\item Definizione struttura di rete
				\item Deploy infrastruttura
				\item Test di sicurezza
				\item Modfica struttura in base alle falle trovate
				\item Deploy struttura finale
				\item Stesura report
			\end{enumerate}
		\item \begin{enumerate}
				\item Brainstorming
				\item Test locali nopCommerce
				\item Revisione manuale file CSV
				\item Selezione architettura Cloud
				\item Installazione locale nopCommerce/ DB su due macchine
				\item Containerazziazione su distro linux
				\item Upload su Cloud Provider 
				\item Calcolo dei prezzi dell'Hosting di tutto il progetto (macchine virtuali, storage, call)
				\item Stesura report economico
			\end{enumerate}
	\end{itemize}
	\section{Strumenti di comunicazione}
		Durante il primo incontro uno dei principali punti che e' stato chiarito e' quello della \emph{comunicazione}.
		E' infatti essenziale che in un gruppo di lavoro sia possibile gestire la comunicazione in maniera piu'
		efficente e inclusiva possibile, senza quindi escludere membri o affidarsi a piattaforme troppo lente o non
		organizzate.
		
		La scelta e' quindi ricaduta sulla piattaforma di messaggistica istantanea \emph{Telegram}, grazie alla
		puntualita' delle opzioni di gestione di una \emph{chat} di gruppo e' possibile \emph{fissare} messaggi, creare
		sondaggi e inviare file di grandi dimensioni. Grazie a recenti aggiornamenti e' inoltre possibile effettuare
		videochiamate e condividere eventualmente il proprio desktop, una feature essenziale nel campo del lavoro
		collaborativo.
	\section{Organizzazione codice}
		Data la forte componente di scrittura software presente all'interno del progetto, si e' propenso per l'utlizzo
		di una piattaforma di sviluppo collaborativo, in maniera da organizzare la stesura del codice nella maniera piu'
		semplice ed esaustiva possibile. In particolare ci siamo affidati al software \emph{GIT} a opera di \emph{Linus
		Torvalds}, creando un organizzazione sulla popolare piattaforma di proprieta' \emph{Microsoft}, \emph{GitHub}.

		Sulla piattaforma ci siamo quindi premurati di creare immediatamente tre \emph{repository} atti a contenere il
		lavoro prodotto dal gruppo, in particolare essi sono:
		\begin{enumerate}
			\item Random\_Script\label{item:Script}
			\item Report\label{item:Report}
			\item Report\_Economy\label{item:ReportE}
		\end{enumerate}

		Il repository numero \ref{item:Script}{, \emph{Random\_Script}} e' atto al contenimento di una serie di
		programmi di piccola entita', come il \emph{parser} che si e' occupato di scaricare le immagini dei prodotti da
		aggiungere successivamente al database dello store, o i \emph{Dockerfile} che serviranno per effettuare l'operazione di \emph{deploy} sull'infrastruttura in ambiente di produzione.

		Per quanto riguarda \emph{Report}, ossia il numero \ref{item:Report}, si tratta dello spazio atto alla creazione
		del report finale, esso e' stato redatto tramite l'utilizzo del linguaggio \LaTeX{}, argomento che sara'
		affrontato in dettaglio in seguito.

		Infine, il \emph{repository} \ref{item:ReportE}, nominato come \emph{Report\_Economy}, e' atto ad accogliere i
		documenti e gli appunti che permetteranno la stesura di un preventivo attendibile dell'implementazione, come
		richiesto dai requisiti del progetto.
	\section{Strumenti di scrittura}
		Come accennato durante la precedente sezione, lo strumento principe che e' stato impiegato per la redazione
		della relazione di progetto e' stato li linguaggio \LaTeX{}. Si tratta di un linguaggio in grado di produrre un
		testo correttamente formattato a partire da semicodice, in questo modo viene automatizzata la gestione di alcune
		importanti caratteristiche del documento finale, come per esempio le immagini, spesso punto di debolezza dei
		comuni software di videoscrittura.
\chapter{Componenti e architettura}
	\section{In generale}
		\subsection{Microsoft Azure}
			Per descrivere l'architettura da noi ideata riteniamo importante descrivere il \emph{provider} al quale
			abbiamo deciso di appoggiarci. La scelta e' ricaduta sullo strumento "\emph{Microsoft Azure}", si tratta
			di un servizio di \emph{Cloud Computing} offerto da \emph{Microsoft}, in particolare esso offre servizi di
			\emph{Platform as a Service}, \emph{Software as a Service} e \emph{Infrastructure as a Service}.
			In particolare e' stato scelto in quanto aderente alle nostre necessita' di gestione di macchine remote da
			parte di un gruppo organizzato, inoltre , \emph{Microsoft} offre un bonus gratuito di credito da spendere
			sulla piattaforma per ogni persona che vi si registri come studente. Cio' in concomitanza con i prezzi in
			linea con il mercato, ci ha permesso di sperimentare senza rischiare di dilapidare denaro.
		\subsection{nopCommerce}
			Durante la nostra ricerca di una soluzione che ci permettesse di creare uno \emph{store online} ci siamo
			imbattuti in \emph{nopCommerce}, una tecnologia a opera di \emph{nopSolutions}. Si tratta di una soluzione
			\href{https://github.com/nopSolutions/nopCommerce/blob/develop/LICENSE.md}{\emph{libre}} e
			\emph{open source} che permette la creazione di \emph{store online} mantanendo una discreta
			semplicita' di utilizzo, nonche una facile integrazione in ogni sistema grazie alla possibilita' di essere
			installato all'interno di \emph{contatiner Docker}.

			Nato nel 2008, sviluppo e supporto non si sono mai interrotti, l'ampia \emph{community} che lo mantiene
			e lo sviluppa permette inoltre una facile risoluzione di eventuali problemi grazie all'ampia
			\href{https://docs.nopcommerce.com/en/developer/index.html?utm\_source=github&utm\_medium=referral&utm\_campaign=documentation&utm\_content=text}{documentazione}
			prodotta nel corso degli anni. Questo prodotto abbraccia le moderne tecnologie in ambito di sviluppo web e
			di sicurezza grazie al massiccio utilizzo di \emph{ASP.NET Core 5} e all'utilizzo come database predefinito
			di \emph{MySQL} fino alle ultime versioni stabili. 
	\section{Architettura di rete}
		L'architettura di rete scelta e' basata sull'utilizzo di una singola macchina virtuale in \emph{cloud} sulla
		sopracitata piattaforma \emph{MS Azure}.

		La macchina virtuale monta un sistema operativo \emph{GNU/Linux Ubuntu Server 20.04 LTS}, e ha le seguenti
		caratteristiche:
		\begin{center}
			\begin{tabular}{c|c}
				Informazione & Metrica \\
				\hline
				Tipologia & Standard\_D2s\_v3 \\
				CPU & 2 \\
				RAM & 8 GB \\
				Disco & 30 GB SSD Premium con ridondanza locale \\
				Sicurezza & Standard \\
			\end{tabular}
		\end{center}

		Al suo interno sono poi presenti alcuni elementi aggiuntivi, che compongono la vera e propria infrastruttura.

		Innanzitutto e' presente \emph{Docker}, si tratta del piu' diffuso orchestratore di \emph{container} diffuso
		sul mercato. I \emph{container} sono "entita'" che al loro interno contengono un ambiente minimale con tutte le
		componenti necessarie a un applicativo per svolgere il suo funzionamento, comprese tutte le dipendenze di ognuna
		delle sue parti. Questa entita' si interfaccia poi con il \emph{Docker Engine}, un software che si occupa di
		tradurre le richieste del container in chiamate al sistema operativo sottostante, in questo caso una
		distribuzione di \emph{Ubuntu GNU/Linux}.

		Da sottolineare poi la fondamentale presenza di \emph{SSH}. Si tratta dell'implementazione dell'omonimo 
		protocollo di connessione remota per sistemi operativi \emph{UNIX like}. Esso permette di effettuare una
		connessione remota con una macchina tramite una coppia di credenziali oppure una chiave \emph{RSA}, criptando
		il traffico in maniera da mantenere la riservatezza della comunicazione.

		Infine, gli ultimi due applicativi che e' opportuno segnalare come parti fondamentali della topografia di rete
		sono \emph{Uncomplicated FireWall} e \emph{Fail2Ban}. Il primo, come intuibile dal nome, e' un \emph{Firewall}
		atto a limitare le connessioni non autorizzate verso la macchina per il quale e' configurato. In particolare,
		questo firewall e' in realta' un \emph{frontend} semplificato per \emph{IP Tables}, probabilmente il piu'
		utilizzato firewall in ambiente \emph{GNU/Linux}.
		Anche \emph{Fail2Ban} si occupa di una funzione simile, in quanto la sua funzione prinicpale all'interno dell'
		architettura proposta e' quella di regolamentare e limitare l'accesso alle connessioni \emph{SSH} per i soggetti
		non autorizzati.
 
	\section{Organizzazione container}
		Come accennato durante il precedente paragrafo, la nostra architettura e' organizzata tramite l'utilizzo di
		\emph{container}. Questa metodologia di  \emph{deploy} e' stata scelta perche offre numerosi vantaggi rispetto
		alla piu' "classica" architettura basata sull'utlizzo di macchine virtuali dedicate. In particolare, una
		macchina e' in grado di gestire diversi \emph{container} diversi, ottimizzando al meglio le risorse
		\emph{hardware} a disposizione. Inoltre, il parziale isolamento di un \emph{container} rispetto alla macchina
		\emph{host} rende l'infrastruttura piu' sicura, in quanto compromettere un singolo applicativo non mette a
		rischio il resto dell'infrastruttura o degli altri servizi che condividono le medesime risorse.
		%%%%%%%%%%%%%%%%%%%%%%%%%%%%%%%%%%%%%%%%%%%%%%%%%%%%%%%%%%%%%%%%%%%%%%%%%%%%%perhaps, non so se lasciarlo o meno
		Infine e' importante sottolineare che, grazie ad alcuni moderni \emph{software} di orchestrazione, come ad
		esempio \emph{Docker swarm}, e' possibile gestire in maniera estremamente puntuale l'eventuale ridondanza dei
		servizi e la loro posizione fisica all'interno del \emph{cluster} di macchine virtuali.
		In questo modo e' possibile rendere un infrastruttura estrememanete resiliente, e' inoltre possibile
		effettuare operazioni di manutenzione con un impatto minimo sull'utente finale.

		Per implementare la nostra infrastruttura, abbiamo deciso di utilizzare i seguenti \emph{container}:
		\begin{itemize}
			\item nopCommerce
			\item MySQL
		\end{itemize}
		Il primo denominato come \emph{nopCommerce} contiene l'effettiva struttura dello \emph{store online}, compreso
		%%%%%%%%%%%%%%%%%%%%%%%%%%%%%%%%%%%%%%%%%%%%%%%ocio alla porta%%%%%%%%%%%%%%%%%%%%%%%%%%%%%%%%%%%%%%%%%%%%%%%%%%
		di tutte le componenti atte a pubblicare le pagine \emph{web} sulla porta \texttt{8080}, tra cui i \emph{CSS} e
		il \emph{backend} scritto in \emph{ASP.NET}.
		
		Il secondo \emph{contaier} invece, contiene un database \emph{MySQL} che si occupera' di contenere tutte le
		informazioni riguardanti i prodotti, ecco alcuni dei campi presenti all'interno del \emph{database}
		\begin{enumerate}
			\item ProductId
			\item ProductType
			\item Name
			\item FullDescription
			\item Vendor
		\end{enumerate}

		I \emph{container} sono configurati in maneira da utilizzare due porte per la comunicazione, esse sono la porta
		\texttt{80} e la porta \texttt{3306}. Rispettivamente usate per permettere al \emph{container} contenente
		\emph{nopCommerce} di essere esposto verso internet, e sempre al medesimo di effettuare le comunicazioni con il database \emph{MySQL}.

\chapter{Processo di implementazione}
	Per gestire al meglio le tempistiche e la messa a terra del progetto abbiamo strutturato un processo diviso in fasi,
	esse ci hanno permesso di poter controllare lo stato di avanzamento dei lavori e modificare i programmi e il carico
	di lavoro in maniera coerente. In particolare possiamo individuare tre principali fasi, elencate di seguito;
	\begin{enumerate}
		\item Brainstorming e ricerca
		\item Prototipazione e scripting
		\item Implementazione locale
		\item Implementazione remota
		\item Test di sicurezza
	\end{enumerate}
	Ognuna di queste fasi del lavoro ha occupato un diverso ruolo e ha necessitato sforzi di natura diversa, ecco dunque
	presentata una breve sintesi di quanto accaduto in ognuna di esse.
	\section{Brainstorming e ricerca}
		Durante la prima parte del progetto, il gruppo si e' dedicato all'individuazione degli strumenti precedentemente
		citati atti all'implementazione di quanto richiesto. Questo iniziale sforzo e' stato portato avanti
		contemporanemente da tutti i membri del gruppo, mentre i membri afferenti al corso di \emph{Cloud Specialist} si
		sono occupati di effettuare le necessarie ricerce per quanto riguarda il provider di servizi, i membri di
		\emph{ICT Security Specialist} si sono occupati di iniziare a definire acluni strumenti atti alla gestione della
		rete.
		Durante questa fase iniziale, sono stati inoltre individuati lo strumento atto alla creazione dello \emph{store
		online} e tutti gli strumenti di comunicazione, organizzazione e videoscrittura.
	\section{Prototipazione e scripting}
		La fase successiva si e' rivalta essere molto piu' pratica di quella appena svolta, in quanto il gruppo si e'
		trovato a dover iniziare a risolvere alcuni problemi pratici, come il \emph{parsing} del file \emph{csv}
		contenente il \emph{database} iniziale e l'utilizzo di \emph{nopCommerce}.

		La necessita' di agire sul file originale contenente i dati iniziali per popolare il \emph{database} e' nata
		dalla scelta operata dal team di sicurezza di mantenere una copia delle immagini localmente alla macchina in
		maniera da ridurre il perimetro di vulnerabilita' dell'infrastruttura. Senza questo passaggio, il
		\emph{database} conterrebbe collegamenti a risorse esterne, i quali potrebbero essere sfruttati da attori
		terzi per ottenere accesso o controllo a parti dell'infrasturttura.

		E' quindi stato prodotto uno \emph{script} tramite  il linguaggio \emph{bash} che, partendo da una trascrizione
		in formato testuale del file originale, possa ottenere le immagini relative ai prodotti conservandone il
		corretto ordine e il riferimento al prodotto. Questo compito e' stato reso piu' difficile da alcuni problemi di
		formattazione conentuti all'interno del file originale. Come prima azione, lo \emph{script} si occupa di
		controllare che tutti gli \emph{URL} siano validi e non nulli, dopodiche procede con l'effettivo scaricamento e
		indicizzazione delle immagini.

		Durante questa fase sono inoltre state esplorate le varie opzioni per l'implementazione di \emph{nopCommerce},
		considerando quale sistema operativo e strategia di \emph{deploy} scegliere tra le diverse disponibli.
	\section{Implementazione locale}
		La fase logicamente successiva ha quindi previsto l'implementazione per intero dello \emph{store} localmente,
		in maniera da poter evidenziare eventuali criticita' e difficolta' di messa in produzione.
	\section{Implementazione remota}%aggiungere anche la fase di testing in remoto
	\section{Test di sicurezza}

\chapter{Penetration testing}
	\chapter{Le premesse}
	\chapter{Information gathering}

\end{document}
