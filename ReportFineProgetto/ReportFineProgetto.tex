\documentclass[a4paper]{report}
\usepackage{graphicx}
\usepackage{eso-pic}
\usepackage{hyperref}
\hypersetup{
	colorlinks=true,
	linkcolor=blue,
	filecolor=magenta,
	urlcolor=blue,
	pdftitle={Report Finale "Laboratorio Integrato"},
	pdfpagemode=FullScreen,
}

\usepackage[italian]{babel}


\newcommand\BackgroundPic{%
	\put(0,0){
		\parbox[b][\paperheight]{\paperwidth}{%
			\vfill
			\centering
			\includegraphics[width=\paperwidth,height=\paperheight,%
			keepaspectratio]{its-cover.pdf}%
			\vfill
		}
	}
}
\author{De lazzari, Dellera, Oglietti,\\
 Murta, Cafasso, Carrieri,\\
 Zuccarella}
\title{Report Finale "Laboratorio Integrato"\\
\large Gruppo 6, Cloud Fiesta}
\date{\today}
\makeindex

\begin{document}
\AddToShipoutPicture*{\BackgroundPic}
\maketitle
\tableofcontents
\chapter{Introduzione}
\author{Riccardo Oglietti} 
	\section{Il Progetto}
	Il qui presente report ha lo scopo di illustrare lo svolgimento nella sua interezza lo svolgimento del progetto a
	opera del gruppo "Cloud Fiesta", il progetto e' stato commissionato dai docenti Blanchietti Andrea e Zimuel Enrico
	nell'ambito del corso \textit{"Laboratorio Integrato"}.

	Lo scopo del progetto e' quello di realizzare una piattaforma di \emph{e-commerce} per conto di un azienda che si
	occupa di commercio al dettaglio, il sistema deve essere \emph{scalabile} in maniera da poter limitare i costi a
	quanto strettamente necessrio e potersi mantenere aderente con le esigenze di crescita aziendale.
	Inoltre, e' essenziale che la piattaforma possa avere degli \emph{standard di sicurezza elevati}, come ben sappiamo,
	durante i recenti anni si e' verificata un impennata dei crimini legati alla \textit{Cybersecurity}, con un
	particolare aumento durante la corrente pandemia da COVID-19, come illustrato
	\href{https://www.interpol.int/en/News-and-Events/News/2020/INTERPOL-report-shows-alarming-rate-of-cyberattacks-during-COVID-19}{dall'Interpol}.
	E' quindi fondamentale che un'applicazione che gestisce flussi di denaro sia quindi estremamente solida dal punto di
	vista della sicurezza informatica.
	Una seconda sezione del progetto, prevede che ogni gruppo si occupi di eseguire dei \emph{penetration test} sul
	gruppo dall'\emph{ID} successivo. Questo per simulare l'ingaggio di un azienda esterna allo scopo di testare la
	sicurezza di un prodotto prima di rilasciarlo effettivamente sul mercato, uno step di decisiva importanza che
	permettera' ai componenti di ogni gruppo di sperimentare le proprie conoscenze di sicurezza informatica all'interno
	di una situazione altamente realistica.

	Vista la complessita' del progetto, e' stato scelto di realizzarlo tramite Team multidisciplinari, con componenti
	appartenenti ad due corsi afferenti agli indirizzi di \emph{Cloud Specialist} e \emph{ICT Security Specialist}.
	All'interno di questi due corsi sono presenti le competenze tecniche atte a svolgere il progetto commissionato,
	coprendo sia l'area di sicurezza e di architettura della rete interna, che quella di utilizzo delle piattaforme
	cloud che permettono di assicurare la scalabilita' necessaria all'azienda.
	\section{Il Team}
	Gli stutenti di entrambi i corsi sono stati divisi in sei differenti gruppi, composti da un totale di otto persone,
	il nostro gruppo, denominato "\emph{Cloud Fiesta}" e' composto dai seguenti studenti:
	\begin{itemize}
		\item \textbf{Cafasso Giovanni}
		\item \textbf{Carrieri Riccardo}
		\item \textbf{De Lazzari Riccardo}
		\item \textbf{Dellera Lorenzo}
		\item \textbf{Murta Alessio}
		\item \textbf{Oglietti Riccardo}
		\item \textbf{Zuccarella Andrea}
	\end{itemize}
	Suddivisi rispettivamente all'interno dei due corsi come da tabella:
	\begin{center}
		\begin{tabular}{c|c}
			Cloud Specialist & ICT Security Specialist \\
			\hline
			Cafasso Giovanni & De Lazzari Riccardo \\
			Carrieri Riccardo & Dellera Lorenzo \\
			Murta Alessio & Oglietti Riccardo \\
			Zuccarella & \\
		\end{tabular}
	\end{center}
	Come consigliato dai docenti, abbiamo assegnato alcuni \emph{ruoli} in grado di aiutarci con l'organizzazione delle
	mansioni e in genere della gestione del progetto, in particolare abbiamo individuato il ruolo di \emph{Team Leader}
	e di e di \emph{Co-Team Leader}, essi sono stati rispettivamente assegnati a \emph{Oglietti Riccardo} e \emph{Murta
	Alessio}. Abbiamo optato per assegnare queste due cariche ripartendole tra i due differenti corsi che compongono il
	gruppo in maniera da manternere un buon livello di equita' e rappresentanza per entrambe le anime del team.

\chapter{Strumenti tecnico-organizzativi}
\author{Oglietti Riccardo}
	\section{GANTT e cronoprogramma}
	Innanzitutto parlando di strumenti tecnico-organizzativi non possiamo che iniziare descrivendo il "\emph{GANTT}.
	Strumento principe per l'organizzazione delle tempistiche, si tratta di una tabella a doppia entrata che permette di
	assegnare alcuni \emph{task} ritenuti fondamentali a un membro e un momento nel quale realizzarlo.

	Ecco una lista riassuntiva dei processi e degli \emph{step} fondamentali che abbiamo individuato al fine della
	realizzazione ottimale del progetto, divisi in base al corso di afferenza dei destinatari:
	\begin{itemize}
		\item \begin{enumerate}
				\item Parsing file CSV
				\item Definizione struttura di rete
				\item Deploy infrastruttura
				\item Test di sicurezza
				\item Modfica struttura in base alle falle trovate
				\item Deploy struttura finale
				\item Stesura report
			\end{enumerate}
		\item \begin{enumerate}
				\item Brainstorming
				\item Test locali nopCommerce
				\item Revisione manuale file CSV
				\item Selezione architettura Cloud
				\item Installazione locale nopCommerce/ DB su due macchine
				\item Containerazziazione su distro linux
				\item Upload su Cloud Provider 
				\item Calcolo dei prezzi dell'Hosting di tutto il progetto (macchine virtuali, storage, call)
				\item Stesura report economico
			\end{enumerate}
	\end{itemize}
	\section{Strumenti di comunicazione}
		Durante il primo incontro uno dei principali punti che e' stato chiarito e' quello della \emph{comunicazione}.
		E' infatti essenziale che in un gruppo di lavoro sia possibile gestire la comunicazione in maniera piu'
		efficente e inclusiva possibile, senza quindi escludere membri o affidarsi a piattaforme troppo lente o non
		organizzate.
		
		La nostra scelta e' quindi ricaduta sulla piattaforma di messaggistica istantanea \emph{Telegram}, grazie alla
		puntualita' delle opzioni di gestione di una \emph{chat} di gruppo e' possibile \emph{pinnare} messaggi, creare
		sondaggi e inviare file di grandi dimensioni. Grazie a recenti aggiornamenti e' inoltre possibile effettuare
		videochiamate e condividere eventualmente il proprio desktop, una feature essenziale nel campo del lavoro
		collaborativo.
	\section{Organizzazione codice}
		Data la forte componente di scrittura software presente all'interno del progetto, abbiamo optato per l'utlizzo
		di una piattaforma di sviluppo collaborativo, in maniera da organizzare la stesura del codice nella maniera piu'
		semplice ed esaustiva possibile. In particolare ci siamo affidati al software \emph{GIT} a opera di \emph{Linus
		Torvalds}, creando un organizzazione sulla popolare piattaforma di proprieta' \emph{Microsoft}, \emph{GitHub}.

		Sulla piattaforma ci siamo quindi premurati di creare immediatamente tre \emph{repository} atti a contenere il
		lavoro da noi prodotto, in particolare essi sono:
		\begin{enumerate}
			\item Random\_Script\label{item:Script}
			\item Report\label{item:Report}
			\item Report\_Economy\label{item:ReportE}
		\end{enumerate}

		Il repository numero \ref{item:Script}{, \emph{Random\_Script}} e' atto al contenimento di una serie di
		programmi di piccola entita', come il \emph{parser} che si e' occupato di scaricare le immagini dei prodotti da
		aggiungere successivamente al database dello store, o i \emph{Dockerfile} che serviranno per \emph{Deployare}
		l'infrastruttura in ambiente di produzione.

		Per quanto riguarda \emph{Report}, ossia il numero \ref{item:Report}

		Infine, il repository \ref{item:ReportE}, nominato come \emph{Report\_Economy}

\end{document}
